\chapter{Pendahuluan}
\label{chap:Pendahuluan}

\section{\textbf{Latar Belakang}}
\label{sec:latar belakang}

Enkripsi adalah proses merahasiakan sebuah informasi dengan cara menyandikan informasi tersebut sehingga informasi tersebut tidak dapat dibaca oleh pihak yang tidak berwenang. Dalam proses enkripsi, dibutuhkan sebuah kunci rahasia (\textit{private key}), untuk menyandikan informasi sehingga tidak bisa dibaca dan untuk mengembalikan informasi sehingga bisa kembali dibaca. Proses enkripsi ini memiliki kelemahan, yaitu jika kunci yang digunakan untuk enkripsi hilang maka  berakibat informasi yang dienkripsi tidak bisa dikembalikan seperti semula.

Pada umumnya, untuk mengembalikan kunci yang hilang ini, beberapa sistem memiliki mekanisme dengan menyediakan sebuah pertanyaan keamanan yang pertanyaan dan jawabannya sudah dirancang oleh pengguna. Jika pengguna menjawab pertanyaan keamanan ini dengan benar maka pengguna bisa mendapatkan kembali kunci yang hilang. Tetapi, 

\section{\textbf{Rumusan Masalah}}
\label{sec:rumusan masalah}

Rumusan masalah pada penelitian ini berupa:
\begin{itemize}
	\item Bagaimana cara melindungi \textit{password} dengan \textit{secret sharing} shamir?
	\item Bagaimana cara mengimplementasikan {\it secret sharing} shamir pada perangkat lunak?
\end{itemize}

\section{\textbf{Tujuan}}
\label{sec:tujuan}

Tujuan penelitian ini berupa:
\begin{itemize}
	\item Mempelajari cara kerja {\it secret sharing} shamir dalam melindungi {\it password}.
	\item Membangun perangkat lunak yang mengimplementasikan {\it secret sharing} shamir
\end{itemize}

\section{\textbf{Batasan Masalah}}
\label{sec:batasan masalah}

Batasan masalah pada penelitian ini berupa:
\begin{itemize}
	\item Setiap pertanyaan selalu dijawab dengan jawaban yang relevan dengan pertanyaan.
\end{itemize}

\section{\textbf{Metodologi Penelitian}}
\label{sec:metodologi penelitian}

Metodologi dalam penelitian ini berupa:
\begin{itemize}
	\item Melakukan studi literatur mengenai {\it secret sharing} shamir
	\item Melakukan studi literatur mengenai algoritma enkripsi \textit{data encryption standard} (DES)
	\item Melakukan studi literatur mengenai \textit{secure-hash-algorithm-512} (SHA-512)
	\item Melakukan analisis dan perancangan mengenai perangkat lunak yang akan dibangun
	\item Implementasi terhadap hasil analisis dan perancangan perangkat lunak
	\item Melakukan pengujian perangkat lunak
\end{itemize}

\section{\textbf{Sistematika Pembahasan}}
\label{sec:sistematika pembahasan}

Sistematika pembahasan dalam penelitian ini berupa:
\begin{itemize}
	\item Bab Pendahuluan
	\\Bab 1 berisi latar belakang, rumusan masalah, tujuan penelitian, batasan masalah, metodologi penelitian, dan sistematika pembahasan.
	\item Bab Dasar Teori
	\\Bab 2 berisi mengenai teori-teori dasar, antara lain kriptografi, algoritma enkripsi, algoritma fungsi \textit{hash}, otentikasi, \textit{secret sharing}, probabilitas, dan entropi.
	\item Bab Analisis
	\\Bab 3 berisi analisis meliputi perhitungan dan proses, \textit{flow chart}, \textit{use case}, dan rancangan awal diagram kelas.
	\item Bab Perancangan
	\\Bab 4 berisi tahapan penjelasan rancangan perangkat lunak meliputi algoritma, diagram kelas lengkap, dan rancangan tampilan perangkat lunak.
	\item Bab Implementasi dan Pengujian
	\\Bab 5 berisi tahapan implementasi pada perangkat lunak meliputi tampilan dari perangkat lunak, pengujian terdiri dari kasus, skenario, dan hasil observasi, dan kesimpulan
	\item Bab Kesimpulan dan Saran
	\\Bab 6 berisi kesimpulan serta beberapa saran untuk pengembangan lebih lanjut dari penelitian yang dilakukan dan perangkat lunak yang dibangun.
\end{itemize}