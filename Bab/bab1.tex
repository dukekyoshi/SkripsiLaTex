\chapter{Pendahuluan}
\label{chap:Pendahuluan}

\section{\textbf{Latar Belakang}}
\label{sec:latar belakang}

Otentikasi adalah proses untuk menentukan apakah sebuah entitas diijinkan untuk mengakses sumber daya yang dimiliki sebuah sistem. Salah satu teknik otentikasi yang paling sederhana dan mudah adalah dengan menggunakan \textit{password}. \textit{Password} ini digunakan dalam proses enkripsi dalam kriptografi. Enkripsi digunakan untuk menjaga keamanan sumber daya yang dimiliki sebuah sistem. \textit{Password} adalah sekumpulan huruf, simbol, dan angka yang sifatnya rahasia. \textit{Password} digunakan bersamaan dengan \textit{username} untuk menandakan bahwa orang yang memiliki \textit{password} dan \textit{username} ini adalah orang yang diberikan ijin untuk mengakses sumber daya yang dimiliki sistem.

Hilangnya \textit{password} dapat menyebabkan sumber daya yang dimiliki oleh sistem tidak dapat diakses, karena itu \textit{password} harus dijaga dengan baik. Namun, hal-hal seperti bencana alam, \textit{human error}, malapetaka lainnya merupakan masalah yang dapat menyebabkan hilangnya password dan masalah-masalah ini tidak dapat dihindari maka diperlukan teknik lain untuk menjaga \textit{password} ini agar tetap aman dan mudah diingat. Salah satu tekniknya dinamakan \textit{secret sharing} shamir. Dalam penelitian ini akan dilakukan analisis bagaimana teknik {\it secret sharing} shamir dapat digunakan untuk melindungi \textit{password}.

\section{\textbf{Rumusan Masalah}}
\label{sec:rumusan masalah}

Rumusan masalah pada penelitian ini berupa:
\begin{itemize}
	\item Bagaimana cara melindungi \textit{password} dengan \textit{secret sharing} shamir?
	\item Bagaimana cara mengimplementasikan {\it secret sharing} shamir pada perangkat lunak?
\end{itemize}

\section{\textbf{Tujuan}}
\label{sec:tujuan}

Tujuan penelitian ini berupa:
\begin{itemize}
	\item Mempelajari cara kerja {\it secret sharing} shamir dalam melindungi {\it password}.
	\item Membangun perangkat lunak yang mengimplementasikan {\it secret sharing} shamir
\end{itemize}

\section{\textbf{Batasan Masalah}}
\label{sec:batasan masalah}

Batasan masalah pada penelitian ini berupa:
\begin{itemize}
	\item Panjang maksimum \textit{password} hanya mencapai 8 karakter.
\end{itemize}

\section{\textbf{Metodologi Penelitian}}
\label{sec:metodologi penelitian}

Metodologi dalam penelitian ini berupa:
\begin{itemize}
	\item Melakukan studi literatur mengenai {\it secret sharing} shamir
	\item Melakukan studi literatur mengenai algoritma enkripsi \textit{data encryption standard} (DES)
	\item Melakukan studi literatur mengenai \textit{secure-hash-algorithm-512} (SHA-512)
	\item Melakukan analisis dan perancangan mengenai perangkat lunak yang akan dibangun
	\item Implementasi terhadap hasil analisis dan perancangan perangkat lunak
	\item Melakukan pengujian perangkat lunak
\end{itemize}

\section{\textbf{Sistematika Pembahasan}}
\label{sec:sistematika pembahasan}

Sistematika pembahasan dalam penelitian ini berupa:
\begin{itemize}
	\item Bab Pendahuluan
	\\Bab 1 berisi latar belakang, rumusan masalah, tujuan penelitian, batasan masalah, metodologi penelitian, dan sistematika pembahasan.
	\item Bab Dasar Teori
	\\Bab 2 berisi mengenai teori-teori dasar, antara lain kriptografi, algoritma enkripsi, algoritma fungsi \textit{hash}, otentikasi, \textit{secret sharing}, probabilitas, dan entropi.
	\item Bab Analisis
	\\Bab 3 berisi analisis terhadap kasus dan perangkat lunak
	\item Bab Perancangan
	\\Bab 4 berisi tahapan penjelasan rancangan perangkat lunak meliputi,
	\item Bab Implementasi dan Pengujian
	\\Bab 5 berisi tahapan implementasi pada perangkat lunak, pengujian yang telah dilakukan terhadap perangkat lunak dan analisis hasil yang diperoleh dari perangkat lunak
	\item Bab Kesimpulan dan Saran
	\\Bab 6 berisi kesimpulan serta beberapa saran untuk pengembangan lebih lanjut dari penelitian yang dilakukan dan perangkat lunak yang dibangun.
\end{itemize}