\chapter{Pendahuluan}
\label{chap:Pendahuluan}

\section{\textbf{Latar Belakang}}
\label{sec:latar belakang}

Otentikasi adalah proses untuk menentukan keaslian identitas dari sebuah entitas saat akan mengakses sumber daya sebuah sistem. Proses otentikasi menentukan apakah sebuah entitas berhak atau tidak untuk mengakses sumber daya sistem tersebut.

Salah satu dari metode otentikasi adalah dengan menggunakan \textit{password}. \textit{Password} adalah sekumpulan huruf, angka, dan simbol yang sifatnya rahasia. Umumnya, \textit{password} digunakan bersamaan dengan \textit{username} untuk mengakses sebuah akun, email, dan sebagainya. Entitas yang memiliki \textit{password} dan \textit{username} diijinkan untuk mengakses akun.

\section{\textbf{Rumusan Masalah}}
\label{sec:rumusan masalah}

Berdasarkan latar belakang yang sudah dibuat, maka permasalahan yang akan dibahas dalam penelitian ini adalah:
\begin{itemize}
	\item Bagaimana mengembalikan \textit{password} untuk banyak akun dengan metode \textit{secret sharing} Shamir?
	\item Bagaimana cara membangun perangkat lunak pengingat \textit{password} yang mengimplementasikan metode \textit{secret sharing} Shamir?
\end{itemize}

\section{\textbf{Tujuan}}
\label{sec:tujuan}

Berdasarkan rumusan masalah yang sudah ditetapkan, maka tujuan dari penelitian ini adalah:
\begin{itemize}
	\item Mempelajari bagaimana metode \textit{secret sharing} Shamir dapat mengembalikan \textit{password} untuk banyak akun.
	\item Membangun perangkat lunak pengingat \textit{password} yang mengimplementasikan metode \textit{secret sharing} Shamir.
\end{itemize}

\section{\textbf{Batasan Masalah}}
\label{sec:batasan masalah}

Batasan masalah pada penelitian ini adalah setiap pertanyaan keamanan dijawab dengan jawaban yang relevan.

\section{\textbf{Metodologi Penelitian}}
\label{sec:metodologi penelitian}

Metodologi dalam penelitian ini berupa:
\begin{itemize}
	\item Melakukan studi literatur untuk mempelajari hal-hal yang diperlukan dalam penggunaan dan implementasi metode \textit{secret sharing} Shamir.
	\item Membangun perangkat lunak yang mengimplementasikan metode \textit{secret sharing} Shamir.
	\item Melakukan pengujian pada perangkat lunak yang sudah dibangun.
\end{itemize}

\section{\textbf{Sistematika Pembahasan}}
\label{sec:sistematika pembahasan}

Sistematika pembahasan dalam penelitian ini berupa:
\begin{itemize}
	\item Bab Pendahuluan
	\\Bab 1 berisi latar belakang, rumusan masalah, tujuan penelitian, batasan masalah, metodologi penelitian, dan sistematika pembahasan.
	\item Bab Dasar Teori
	\\Bab 2 berisi mengenai teori-teori dasar, antara lain kriptografi, algoritma enkripsi, algoritma fungsi \textit{hash}, otentikasi, \textit{secret sharing}, probabilitas, dan entropi.
	\item Bab Analisis
	\\Bab 3 berisi analisis meliputi perhitungan dan proses, \textit{flow chart}, \textit{use case}, dan rancangan awal diagram kelas.
	\item Bab Perancangan
	\\Bab 4 berisi tahapan penjelasan rancangan perangkat lunak meliputi algoritma, diagram kelas lengkap, dan rancangan tampilan perangkat lunak.
	\item Bab Implementasi dan Pengujian
	\\Bab 5 berisi tahapan implementasi pada perangkat lunak meliputi tampilan antarmuka perangkat lunak, pengujian perangkat lunak, dan kesimpulan.
	\item Bab Kesimpulan dan Saran
	\\Bab 6 berisi kesimpulan serta beberapa saran untuk pengembangan lebih lanjut dari penelitian yang dilakukan dan perangkat lunak yang dibangun.
\end{itemize}